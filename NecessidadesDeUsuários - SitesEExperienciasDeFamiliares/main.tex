\documentclass{article}
\usepackage[utf8]{inputenc}
\usepackage[brazil]{babel} 
\usepackage{placeins}
\title{IHC}
\author{danny.menezes }
\date{April 2019}

\usepackage{natbib}
\usepackage{graphicx}

\begin{document}



\section{WebSites para compra de ingressos}
    WebSites oferece ao consumidor simplicidade, agilidade e comodidade na hora de comprar um ingresso (sem demora e sem filas), então analisamos o funcionamento de alguns WebSites de vendas de ingresso de cinema.

	Foi feita uma pesquisa no google onde encontramos sites de venda, o primeiro site escolhido foi o que apareceu em primeiro lugar em forma de publicidade: \cite{letsEvents}.
	
	Se chama Lets.events, nele além de comprar ingressos, você pode também criar eventos para vender seus ingressos. A primeira vista é um site receptivo com imagens coloridas e bastante informação de como comprar incluindo valores de taxas que você irá pagar na compra, junto do valor da taxa há um especificação de que naquela site você encontra as menores taxas do mercado. 
    No site também há uma plataforma de atendimento a pessoa que deseja criar eventos.
    Quando se clica em encontrar eventos, logo carrega uma página com eventos recentes e campo para procura do que você deseja. Digitei cinema e foi retornado, apenas um evento, logo podemos notar que não se trata de cinema em si (com salas e telas), mas sim eventos culturais que acontecem em locais diferentes e até mesmo ao ar livre. 
    Ponto positivo é que há toda uma especificação do local, horário, valores, organizadores, enfim, bem completo. Na hora da compra você escolhe a quantidade de ingressos e se é meia entrada com promoção, insere seus dados e realiza a compra. Para finalizar é apenas imprimir seu ingresso.

    Na segunda pesquisa parece que houve um erro no site que logo mais explicarei sobre. O segundo site estava entre as pesquisas do topo, se trata do site \cite{Cinemark}. É um site do próprio cinema, sua apresentação é simples e sem muitos detalhes chamativos, mas contém todas as informações necessárias. Uma elemento que chamou a atenção foi a venda de ingressos corporativos que fornece uma promoção para aqueles que querem presentear a equipe de funcionários de uma empresa com uma sessão de cinema que acompanha um brinde. Isso é uma forma atrativa e criativa de venda de ingressos, possibilitando assim que mais pessoas tenham acesso ao cinema. Como dito anteriormente, quando acessado o E-COMMERCE para a compra de ingressos corporativos teve uma uma demora de aproximadamente 15 minutos para que carregasse a página, o que é um ponto negativo para o site.
    Na busca por ingressos, há uma apresentação dos shoppings que possuem o Cinemark, para que o cliente possa escolher a melhor localização, há também a busca por ingressos do filme que você deseja assistir. 
    Os filmes disponíveis estão em programação, lá você pode escolher horários, filmes e localização. Ponto positivo é que há cardápio de lanchonete do cinema e também promoções de ingressos para clientes assíduos.
    É um site completo que com certeza nos ajudaria na compra de ingressos.

	Terceiro site visitado é um bem comum \cite{ingresso.com}, abrange muitos estados do Brasi, onde você pode escolher sua localização. Na página inicial já aparece todos os filmes em cartaz e o mais procurado é evidenciado no topo.
	Tem a opção de escolha de filmes, cinema e eventos para compra de ingressos, conta também com atendimento ao consumidor.
	Ao clicar no filme desejado aparece todos os cinemas na cidade escolhida, bem como os horários de transmissão e em qual horário possui o filme, em 3D se é dublado ou legendado e a sala dos filmes, possuindo também o guia de salas.Na hora da compra você escolhe as poltronas disponíveis e poltronas especiais, o que é um ponto positivo na organização de um lugar e se deseja adicionar pipoca ao valor do ingresso na hora de pagar. 
	Finalizando você escolhe a forma de pagamento e imprime seu ingresso. Por fim, notamos que é um site completo e útil que com certeza iremos recomendar.
	
	\section{Experiências de familiares e amigos sobre compra de ingresso de cinema }
	Melhor maneira encontrada de lidar e organizar pesquisas foi criando um questionário objetivo, pois é rápido e eficaz. Foram entrevistadas 11 pessoas, tanto parentes como amigos próximos com intuito de sabermos suas experiências, preferências e até mesmo o que para eles seriam importante na hora da compra de um ingresso de cinema.
    Será apresentado agora tabelas com os dados coletados.

              \begin{table}[h]
            \centering
            \resizebox{\textwidth}{!}{
                \begin{tabular}{|c|c|c|c|c|c|c|c|c|c|c|c|}
                    \hline
                    \multicolumn{1}{|c|}{\textbf{}} & 
                    \multicolumn{1}{c|}{\textbf{Carlos}} & 
                    \multicolumn{1}{c|}{\textbf{Danilo}} & 
                    \multicolumn{1}{c|}{\textbf{Gabriel}} & 
                    \multicolumn{1}{c|}{\textbf{Giovani}} & 
                    \multicolumn{1}{c|}{\textbf{Giovanna}} & 
                    \multicolumn{1}{c|}{\textbf{Guilherme}} & 
                    \multicolumn{1}{c|}{\textbf{Indianara}} &
                    \multicolumn{1}{c|}{\textbf{Luiz}} &
                    \multicolumn{1}{c|}{\textbf{Micael}} &
                    \multicolumn{1}{c|}{\textbf{Nicole}} &
                    \multicolumn{1}{c|}{\textbf{Vanessa}} \\ \hline
                    Em casa & & X & & & & & X & & X & X & X \\ \hline
                    No cinema & & & X & X & X & X & & X & & & \\ \hline
                \end{tabular}
            }
            \caption{Preferência de local para ver filme}
            \label{table:preferencia_local_ver_filme}
        \end{table}
        \FloatBarrier
    
     Na Tabela \ref{table:preferencia_local_ver_filme} podemos notar que maioria dos entrevistados preferem ver um filme no cinema, mas a disputa ficou em acirrada entre os que preferem o conforto de casa ou invés da descontração e tela gigante do cinema.

          \begin{table}[h]
            \centering
            \resizebox{\textwidth}{!}{
                \begin{tabular}{|c|c|c|c|c|c|c|c|c|c|c|c|c|}
                    \hline
                    \multicolumn{1}{|c|}{\textbf{}} & 
                    \multicolumn{1}{c|}{\textbf{Carlos}} & 
                    \multicolumn{1}{c|}{\textbf{Danilo}} & 
                    \multicolumn{1}{c|}{\textbf{Gabriel}} & 
                    \multicolumn{1}{c|}{\textbf{Giovani}} & 
                    \multicolumn{1}{c|}{\textbf{Giovanna}} & 
                    \multicolumn{1}{c|}{\textbf{Guilherme}} & 
                    \multicolumn{1}{c|}{\textbf{Indianara}} &
                    \multicolumn{1}{c|}{\textbf{Luiz}} &
                    \multicolumn{1}{c|}{\textbf{Micael}} &
                    \multicolumn{1}{c|}{\textbf{Nicole}} &
                    \multicolumn{1}{c|}{\textbf{Vanessa}} &
                    \multicolumn{1}{c|}{\textbf{Média}}\\ \hline
                    Alimentação Disponível & 5 & 10 & 3 & 1 & 0 & 6 & 5 & 6 & 10 & 5 & 8 & 5.4\\ \hline
                    Preço do ingresso & 8 & 9 & 8 & 10 & 10 & 8 & 10 & 9 & 10 & 10 & 10 & 9.2\\ \hline
                    Localização & 8 & 5 & 5 & 6 & 10 & 6 & 5 & 6 & 10 & 8 & 7 & 6.9\\ \hline
                    Ambiente confortável  & 9 & 7 & 8 & 7 & 9 & 9 & 9 & 6 & 10 & 9 & 9 & 8.3\\ \hline
                    Disponibilidade de horário  & 9 & 4 & 8 & 8 & 7 & 9 & 8 & 9 & 10 & 10 & 8 & 8.1\\ \hline
                \end{tabular}
            }
            \caption{Representação de 1 a 10 na hora de escolher um cinema}
            \label{table:esolha_de_cinema}
        \end{table}
        \FloatBarrier
     Feito a média de cada tópico levantado na Tabela \ref{table:esolha_de_cinema}, ficou claro que a maioria dos entrevistados estão mesmo preocupados mais com o preço do ingresso na hora de adquirir um e que não importa muito para alguns que haja comida no estabelecimento.

              \begin{table}[h]
            \centering
            \resizebox{\textwidth}{!}{
                \begin{tabular}{|c|c|c|c|c|c|c|c|c|c|c|c|}
                    \hline
                    \multicolumn{1}{|c|}{\textbf{}} & 
                    \multicolumn{1}{c|}{\textbf{Carlos}} & 
                    \multicolumn{1}{c|}{\textbf{Danilo}} & 
                    \multicolumn{1}{c|}{\textbf{Gabriel}} & 
                    \multicolumn{1}{c|}{\textbf{Giovani}} & 
                    \multicolumn{1}{c|}{\textbf{Giovanna}} & 
                    \multicolumn{1}{c|}{\textbf{Guilherme}} & 
                    \multicolumn{1}{c|}{\textbf{Indianara}} &
                    \multicolumn{1}{c|}{\textbf{Luiz}} &
                    \multicolumn{1}{c|}{\textbf{Micael}} &
                    \multicolumn{1}{c|}{\textbf{Nicole}} &
                    \multicolumn{1}{c|}{\textbf{Vanessa}} \\ \hline
                    Aplicativos & & & & & & & & & & & \\ \hline
                    Sites & & & & & & X & & & & X & \\ \hline
                    Presencial & X & X & X & X & X & & X & X & X & & X\\ \hline
                \end{tabular}
            }
            \caption{Meio mais frequente de compra de ingresso}
            \label{table:meio_freq_comp_de_ingresso}
        \end{table}
        \FloatBarrier
     Vendas de ingresso em sites e por aplicativos perdem para a presencial, isso é um ponto que deve ser abordado de como melhorar as vendas já que é mais cômodo comprar pela internet.

           \begin{table}[h]
            \centering
            \resizebox{\textwidth}{!}{
                \begin{tabular}{|c|c|c|c|c|c|c|c|c|c|c|c|}
                    \hline
                    \multicolumn{1}{|c|}{\textbf{}} & 
                    \multicolumn{1}{c|}{\textbf{Carlos}} & 
                    \multicolumn{1}{c|}{\textbf{Danilo}} & 
                    \multicolumn{1}{c|}{\textbf{Gabriel}} & 
                    \multicolumn{1}{c|}{\textbf{Giovani}} & 
                    \multicolumn{1}{c|}{\textbf{Giovanna}} & 
                    \multicolumn{1}{c|}{\textbf{Guilherme}} & 
                    \multicolumn{1}{c|}{\textbf{Indianara}} &
                    \multicolumn{1}{c|}{\textbf{Luiz}} &
                    \multicolumn{1}{c|}{\textbf{Micael}} &
                    \multicolumn{1}{c|}{\textbf{Nicole}} &
                    \multicolumn{1}{c|}{\textbf{Vanessa}} \\ \hline
                    Aplicativos & & & & & & & & & & & \\ \hline
                    Sites & X & X & X & X & X & X & X & X & X & X & X \\ \hline
                    Cartazes & & & & & & & & & & & \\ \hline
                \end{tabular}
            }
            \caption{Meio de obter informações antes de comprar um ingresso}
            \label{table:obter_inf}
        \end{table}
        \FloatBarrier
    
    Visto na Tabela \ref{table:obter_inf}, informações são sempre procuradas em sites.
    
           \begin{table}[h]
            \centering
            \resizebox{\textwidth}{!}{
                \begin{tabular}{|c|c|c|c|c|c|c|c|c|c|c|c|}
                    \hline
                    \multicolumn{1}{|c|}{\textbf{}} & 
                    \multicolumn{1}{c|}{\textbf{Carlos}} & 
                    \multicolumn{1}{c|}{\textbf{Danilo}} & 
                    \multicolumn{1}{c|}{\textbf{Gabriel}} & 
                    \multicolumn{1}{c|}{\textbf{Giovani}} & 
                    \multicolumn{1}{c|}{\textbf{Giovanna}} & 
                    \multicolumn{1}{c|}{\textbf{Guilherme}} & 
                    \multicolumn{1}{c|}{\textbf{Indianara}} &
                    \multicolumn{1}{c|}{\textbf{Luiz}} &
                    \multicolumn{1}{c|}{\textbf{Micael}} &
                    \multicolumn{1}{c|}{\textbf{Nicole}} &
                    \multicolumn{1}{c|}{\textbf{Vanessa}} \\ \hline
                    Frequentemente & & & & & & & & & & & \\ \hline
                    Raramente & X & X & X & X & X & X & X & X & & X & X \\ \hline
                    Nunca & & & & & & & & & X & & \\ \hline
                \end{tabular}
            }
            \caption{Frenquência em que os entrevistados veem anúncios de venda de ingressos de cinema}
            \label{table:freq_anuncios}
        \end{table}
        \FloatBarrier
    Encontramos uma deficiência que pode influenciar na perda de pessoas que poderiam criar um interesse em comprar ingressos através da publicidade, visto que é raro a divulgação e promoção de venda. 
    
           \begin{table}[h]
            \centering
            \resizebox{\textwidth}{!}{
                \begin{tabular}{|c|c|c|c|c|c|c|c|c|c|c|c|}
                    \hline
                    \multicolumn{1}{|c|}{\textbf{}} & 
                    \multicolumn{1}{c|}{\textbf{Carlos}} & 
                    \multicolumn{1}{c|}{\textbf{Danilo}} & 
                    \multicolumn{1}{c|}{\textbf{Gabriel}} & 
                    \multicolumn{1}{c|}{\textbf{Giovani}} & 
                    \multicolumn{1}{c|}{\textbf{Giovanna}} & 
                    \multicolumn{1}{c|}{\textbf{Guilherme}} & 
                    \multicolumn{1}{c|}{\textbf{Indianara}} &
                    \multicolumn{1}{c|}{\textbf{Luiz}} &
                    \multicolumn{1}{c|}{\textbf{Micael}} &
                    \multicolumn{1}{c|}{\textbf{Nicole}} &
                    \multicolumn{1}{c|}{\textbf{Vanessa}} \\ \hline
                    Sim & X & X & X & & & X & X & X & X & X & X \\ \hline
                    Não & & & & X & X & & & & & & \\ \hline
                \end{tabular}
            }
            \caption{Propagantas geralmente está relacionada ao gosto do entrevistado?}
            \label{table:gosto_do_entrevistado}
        \end{table}
        \FloatBarrier
    Um ponto positivo é que as raras propagandas estão relacionadas com os gostos dos entrevistados.
    
           \begin{table}[h]
            \centering
            \resizebox{\textwidth}{!}{
                \begin{tabular}{|c|c|c|c|c|c|c|c|c|c|c|c|}
                    \hline
                    \multicolumn{1}{|c|}{\textbf{}} & 
                    \multicolumn{1}{c|}{\textbf{Carlos}} & 
                    \multicolumn{1}{c|}{\textbf{Danilo}} & 
                    \multicolumn{1}{c|}{\textbf{Gabriel}} & 
                    \multicolumn{1}{c|}{\textbf{Giovani}} & 
                    \multicolumn{1}{c|}{\textbf{Giovanna}} & 
                    \multicolumn{1}{c|}{\textbf{Guilherme}} & 
                    \multicolumn{1}{c|}{\textbf{Indianara}} &
                    \multicolumn{1}{c|}{\textbf{Luiz}} &
                    \multicolumn{1}{c|}{\textbf{Micael}} &
                    \multicolumn{1}{c|}{\textbf{Nicole}} &
                    \multicolumn{1}{c|}{\textbf{Vanessa}} \\ \hline
                    Sites & & & & & & X & & & & X & \\ \hline
                    Presencial & & X & & X & X & & X & & X & & X \\ \hline
                    AutoAtendimento & X & & & & & & & & & & \\ \hline
                    Todas & & & X & & & & & X & & & \\ \hline
                \end{tabular}
            }
            \caption{Baseado nas 3 últimas visitas, essas foram a forma de compra de ingressos dos intrevistados}
            \label{table:forma_de_compra}
        \end{table}
        \FloatBarrier
     Se nota na tabela \ref{table:forma_de_compra} que ficou bem dividido os meios de compra de ingressos entre os entrevistados de acordo com suas necessidades e conforto.
     
             \begin{table}[h]
            \centering
            \resizebox{\textwidth}{!}{
                \begin{tabular}{|c|c|c|c|c|c|c|c|c|c|c|c|c|}
                    \hline
                    \multicolumn{1}{|c|}{\textbf{}} & 
                    \multicolumn{1}{c|}{\textbf{Carlos}} & 
                    \multicolumn{1}{c|}{\textbf{Danilo}} & 
                    \multicolumn{1}{c|}{\textbf{Gabriel}} & 
                    \multicolumn{1}{c|}{\textbf{Giovani}} & 
                    \multicolumn{1}{c|}{\textbf{Giovanna}} & 
                    \multicolumn{1}{c|}{\textbf{Guilherme}} & 
                    \multicolumn{1}{c|}{\textbf{Indianara}} &
                    \multicolumn{1}{c|}{\textbf{Luiz}} &
                    \multicolumn{1}{c|}{\textbf{Micael}} &
                    \multicolumn{1}{c|}{\textbf{Nicole}} &
                    \multicolumn{1}{c|}{\textbf{Vanessa}} &
                    \multicolumn{1}{c|}{\textbf{Média}}\\ \hline
                     & 10 & 4 & 8 & 0 & 0 & 8 & 7 & 8 & 0 & 10 & 9 & 5.8\\ \hline
                \end{tabular}
            }
            \caption{De 1 a 10 qual a possibilidade do entrevistado indicar o último site de compra de ingresso de cinema}
            \label{table:indicar_site_compra}
        \end{table}
        \FloatBarrier
    
    A opção de informação em sites vista anteriormente ganhou de acordo com a Tabela \ref{table:obter_inf}, mas ela perde quando se trata de compra de ingresso visto na Tabela \ref{table:meio_freq_comp_de_ingresso}, na Tabela \ref{table:indicar_site_compra} a média de indicação de site de venda é de 5,8, isso poderia ser melhorado, pois se houver indicação de uma pessoa a outra o poder de venda aumenta muito, trazendo lucro para as empresas.
   
           \begin{table}[h]
            \centering
            \resizebox{\textwidth}{!}{
                \begin{tabular}{|c|c|c|c|c|c|c|c|c|c|c|c|}
                    \hline
                    \multicolumn{1}{|c|}{\textbf{}} & 
                    \multicolumn{1}{c|}{\textbf{Carlos}} & 
                    \multicolumn{1}{c|}{\textbf{Danilo}} & 
                    \multicolumn{1}{c|}{\textbf{Gabriel}} & 
                    \multicolumn{1}{c|}{\textbf{Giovani}} & 
                    \multicolumn{1}{c|}{\textbf{Giovanna}} & 
                    \multicolumn{1}{c|}{\textbf{Guilherme}} & 
                    \multicolumn{1}{c|}{\textbf{Indianara}} &
                    \multicolumn{1}{c|}{\textbf{Luiz}} &
                    \multicolumn{1}{c|}{\textbf{Micael}} &
                    \multicolumn{1}{c|}{\textbf{Nicole}} &
                    \multicolumn{1}{c|}{\textbf{Vanessa}} \\ \hline
                    Semanalmente & & & & & & & & & & & \\ \hline
                    Mais de uma vez ao mês & & & & & & & & & & & \\ \hline
                    Mensalmente & & & & & & & & & & & \\ \hline
                    Lançamento imperdível & X & X & X & X & X & X & X & X & & X & \\ \hline
                    Nenhuma das anteriores & & & & & & & & & X & & X \\ \hline
                \end{tabular}
            }
            \caption{Frequência que o entrevistado vai ao cinema }
            \label{table:freq_cinema}
        \end{table}
        \FloatBarrier
     O que precisa ser feito para aumentar o número de vezes em que as pessoas vão ao cinema? Essa pergunta se responde com base na tabela \ref{table:esolha_de_cinema}, pois atualmente os ingressos estão com preços altos e de acordo com os entrevistados a importância do preço acessível chega a 9,2. Também procuramos saber o que mais estaria impedindo esses entrevistados de irem ao cinema e notamos que a localização está acima da média em questão de importância, pois a maioria deles moram em cidades onde não há cinemas. 

              \begin{table}[h]
            \centering
            \resizebox{\textwidth}{!}{
                \begin{tabular}{|c|c|c|c|c|c|c|c|c|c|c|c|}
                    \hline
                    \multicolumn{1}{|c|}{\textbf{}} & 
                    \multicolumn{1}{c|}{\textbf{Carlos}} & 
                    \multicolumn{1}{c|}{\textbf{Danilo}} & 
                    \multicolumn{1}{c|}{\textbf{Gabriel}} & 
                    \multicolumn{1}{c|}{\textbf{Giovani}} & 
                    \multicolumn{1}{c|}{\textbf{Giovanna}} & 
                    \multicolumn{1}{c|}{\textbf{Guilherme}} & 
                    \multicolumn{1}{c|}{\textbf{Indianara}} &
                    \multicolumn{1}{c|}{\textbf{Luiz}} &
                    \multicolumn{1}{c|}{\textbf{Micael}} &
                    \multicolumn{1}{c|}{\textbf{Nicole}} &
                    \multicolumn{1}{c|}{\textbf{Vanessa}} \\ \hline
                    Sim & X & & & & & X & & & & & \\ \hline
                    Não & & X & X & X & X & & X & X & X & X & X \\ \hline
                \end{tabular}
            }
            \caption{Tabela sobre se o entrevistado usa aplicativo de celular para compra de ingresso de cinema}
            \label{table:Aplicativo_cell}
        \end{table}
     Dois dos entrevistados possuem aplicativos para compra de ingresso, mas já era de se esperar, pois a frequência dos entrevistados em cinema é baixa.

     \begin{table}[h]
        \centering
        \resizebox{\textwidth}{!}{
            \begin{tabular}{|c|c|c|c|c|c|c|c|c|c|c|c|}
                \hline
                \multicolumn{1}{|c|}{\textbf{}} & 
                \multicolumn{1}{c|}{\textbf{Carlos}} & 
                \multicolumn{1}{c|}{\textbf{Danilo}} & 
                \multicolumn{1}{c|}{\textbf{Gabriel}} & 
                \multicolumn{1}{c|}{\textbf{Giovani}} & 
                \multicolumn{1}{c|}{\textbf{Giovanna}} & 
                \multicolumn{1}{c|}{\textbf{Guilherme}} & 
                \multicolumn{1}{c|}{\textbf{Indianara}} &
                \multicolumn{1}{c|}{\textbf{Luiz}} &
                \multicolumn{1}{c|}{\textbf{Micael}} &
                \multicolumn{1}{c|}{\textbf{Nicole}} &
                \multicolumn{1}{c|}{\textbf{Vanessa}} \\ \hline
                Sim  &   & X & X & X & X & X & X & X & X &   &  X\\ \hline
                Não  & X &   &   &   &   &   &   &   &   & X &   \\ \hline
            \end{tabular}
        }
        \caption{O AutoAtendimento no cinema é útil?}
        \label{table:Utilidade_autoatendimento}
        \end{table}
        \FloatBarrier
    Praticidade é bem vinda entre a maioria dos entrevistados.

    \begin{table}[h]
        \centering
        \resizebox{\textwidth}{!}{
            \begin{tabular}{|c|c|c|c|c|c|c|c|c|c|c|c|}
                \hline
                \multicolumn{1}{|c|}{\textbf{}} & 
                \multicolumn{1}{c|}{\textbf{Carlos}} & 
                \multicolumn{1}{c|}{\textbf{Danilo}} & 
                \multicolumn{1}{c|}{\textbf{Gabriel}} & 
                \multicolumn{1}{c|}{\textbf{Giovani}} & 
                \multicolumn{1}{c|}{\textbf{Giovanna}} & 
                \multicolumn{1}{c|}{\textbf{Guilherme}} & 
                \multicolumn{1}{c|}{\textbf{Indianara}} &
                \multicolumn{1}{c|}{\textbf{Luiz}} &
                \multicolumn{1}{c|}{\textbf{Micael}} &
                \multicolumn{1}{c|}{\textbf{Nicole}} &
                \multicolumn{1}{c|}{\textbf{Vanessa}} \\ \hline
                Sozinho  &   &   &   & X &   &   &   &   &   &   & \\ \hline
                Em dupla & X &   &   &   &   &   &   & X &   &   & \\ \hline
                Em grupo &   & X & X &   & X & X & X &   & X & X & \\ \hline
            \end{tabular}
        }
        \caption{Como você vai ao cinema na maioria das vezes}
        \label{table:Companhia_cinema}
        \end{table}
        \FloatBarrier
     
     Um ponto positivo na venda de ingressos é que geralmente os entrevistados vão em grupo de amigos ou em par, o que aumenta a visibilidade do cinema e como um ponto alternativo de lazer entre várias pessoas.

    
    
 \nocite{adams1995hitchhiker}

\bibliographystyle{IEEEtran}
\bibliography{references}
\end{document}
